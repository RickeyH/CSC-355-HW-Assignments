\section{Question 3}

\begin{question}
    Implement the Shamanskii method for nonlinear systems in \MATLAB in a function of the form \verb+function p = sham_sys(F,J,p0,tol,maxiter)+. Start with a Newton step. Test it on both F and G. How do the final solutions and number of iterations compare to Newton’s method?
\end{question}

\begin{answer}
    I first implemented the Shamanskii method in \MATLAB using the following code
    \begin{verbatim}
        %% The function sham_sys
        % The input are system of equations F, Jacobian J, the initial guess 
        % p0, tolerance tol and the max number of iteration maxiter. The output is
        % a vector of approximation p.
        function p = sham_sys(F,J,p0,tol,maxiter)
            i = 1;
            while(i <= maxiter)
                if(mod(i,2) == 1)
                    q = J(p0);
                    y = q\(-F(p0));
                else
                    y = q\(-F(p0));
                end
                x = p0 + y;
                if (norm(y,2) < tol)
                    break
                end
                p0 = x;
                p(:,i) = p0;
                i = i + 1;
            end
        end
    \end{verbatim}
    Then, I test the function on $F$ and $G$ with the same initial value, maximal number of iteration, and tolerance. The code I used is shown below:
    \begin{verbatim}
        %% Question 3
        % function and Jacobian, assuming x is a vector with 3 components
        F = @(x) [3*x(1)-cos(x(2)*x(3))-.5;
            (x(1))^2-625*(x(2))^2-0.25;
            exp(-x(1)*x(2))+20*x(3)+(10*pi-3)/3];
        J = @(x) [3, x(3)*sin(x(2)*x(3)), x(2)*sin(x(2)*x(3));
            2*x(1), -1250*x(2), 0;
            -x(2)*exp(-x(1)*x(2)), -x(1)*exp(-x(1)*x(2)), 20];
        
        % intial guess
        x0 = [0.1;0.1;-0.1];
        
        % set other parameters
        tol = 10e-10;
        maxiter = 50;
        
        % call newton_sys function here
        p = sham_sys(F,J,x0,tol,maxiter);
        %% another input
        % function and Jacobian, assuming x is a vector with 3 components
        G = @(x) [3*x(1)-cos(x(2)*x(3))-.5;
            (x(1))^2-81*(x(2)+0.1)^2+sin(x(3))+1.06;
            exp(-x(1)*x(2))+20*x(3)+(10*pi-3)/3];
        J2 = @(x) [3, x(3)*sin(x(2)*x(3)), x(2)*sin(x(2)*x(3));
            2*x(1), -162*(x(2)+0.1), cos(x(3));
            -x(2)*exp(-x(1)*x(2)), -x(1)*exp(-x(1)*x(2)), 20];
        
        % intial guess
        x0 = [0.1;0.1;-0.1];
        
        % set other parameters
        tol = 10e-10;
        maxiter = 50;
        
        % call newton_sys function here
        p2 = sham_sys(G,J2,x0,tol,maxiter);
    \end{verbatim}
    Then for the input $F$, I got a result of
    \begin{equation}
        \left[\begin{matrix}
            0.500000000000000\\
            2.81952089189883\cdot 10^{-9}\\
            -0.523598775527811
        \end{matrix}\right]
    \end{equation}
    , and it took $35$ iterations to get this result, while for the input $G$, the function gave result of 
    \begin{equation}
        \left[\begin{matrix}
            0.500000000000010\\
            1.12406669361333\cdot 10^{-12}\\
            -0.523598775598270
        \end{matrix}\right]
    \end{equation}
    , and it took only $6$ iterations to get this result.
    
    Comparing to the results from the Newton's method, the second entries of the results are different, and in general, the more iterations were taken using the Shamanshii method.
\end{answer}