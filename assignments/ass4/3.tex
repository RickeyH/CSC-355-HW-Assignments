\section{Question 3}

\begin{question}
    Derive the general system of equations that can be solved to obtain a cubic spline made up of $n$ cubic polynomials $S_j(x) = a_j+b_j(x-x_j)+c_j(x-x_j)^2+d_j(x-x_j)^3$ for a given function $f$ over an interval divided into $n$ subintervals by the points $x_0, x_1, x_2, \dots, x_n$. Use the not-a-knot boundary conditions.
    
    (not-a-knot condition: ${s_0}^{'''}(x_1) = {s_1}^{'''}(x_1)$; ${s_n}^{'''}(x_{n-1}) = {s_{n-1}}^{'''}(x_{n-1})$)
\end{question}

\begin{answer}
    We will use the following $n$ polynomials to derive the general system of equations:
    \begin{align}
        S_0(x) &= a_0 + b_0(x-x_0) + c_0(x-x_0)^2 + d_0(x-x_0)^3\\
        S_1(x) &= a_1 + b_1(x-x_1) + c_1(x-x_1)^2 + d_1(x-x_1)^3\\
        &\vdots\\
        S_{n-2}(x) &= a_{n-2} + b_{n-2}(x-x_{n-2}) + c_{n-2}(x-x_{n-2})^2 + d_{n-2}(x-x_{n-2})^3\\
        S_{n-1}(x) &= a_{n-1} + b_{n-1}(x-x_{n-1}) + c_{n-1}(x-x_{n-1})^2 + d_{n-1}(x-x_{n-1})^3
    \end{align}
    Then, these polynomials need to satisfy the following conditions:
    \begin{align}
        S_0(x_0) = f(x_0) &\Rightarrow a_0 = f(x_0)\\
        S_1(x_1) = f(x_1) &\Rightarrow a_1 = f(x_1)\\
        &\vdots\\
        S_{n-2}(x_{n-2}) = f(x_{n-2}) &\Rightarrow a_{n-2} = f({x_{n-2}})\\
        S_{n-1}(x_{n-1}) = f(x_{n-1}) &\Rightarrow a_{n-1} = f({x_{n-1}})\\
        &\newline\\
        S_0(x_1) = f(x_1) &\Rightarrow a_0 + (x_1-x_0)b_0 + {(x_1-x_0)}^2c_0 + {(x_1-x_0)}^3d_0 = f(x_1)\\
        S_1(x_2) = f(x_2) &\Rightarrow a_1 + (x_2-x_1)b_1 + {(x_2-x_1)}^2c_1 + {(x_2-x_1)}^3d_1 = f(x_2)\\
        &\vdots\\
        S_{n-2}(x_{n-1}) = f(x_{n-1}) &\Rightarrow a_{n-2} + (x_{n-1} - x_{n-2})b_{n-2} + {(x_{n-1}-x_{n-2})}^2c_{n-2}\\
        & \;\;\;\; + {(x_{n-1} - x_{n-2})}^3d_{n-2} = f(x_{n-1})\\
        S_{n-1}(x_n) = f(x_n) &\Rightarrow a_{n-1} + (x_n - x_{n-1})b_{n-1} + {(x_n-x_{n-1})}^2c_{n-1}\\
        & \;\;\;\; + {(x_n - x_{n-1})}^3d_{n-1} = f(x_n)
    \end{align}
    Hence, This gives us $2n$ equations.
    
    Computing the first and second derivatives of the polynomials, we have
    \begin{align}
        {S_0}^{'}(x) &= b_0 + 2c_0(x-x_0) + 3d_0(x-x_0)^2\\
        {S_1}^{'}(x) &= b_1 + 2c_1(x-x_1) + 3d_1(x-x_1)^2\\
        &\vdots\\
        {S_{n-2}}^{'}(x) &= b_{n-2} + 2c_{n-2}(x-x_{n-2}) + 3d_{n-2}(x-x_{n-2})^2\\
        {S_{n-1}}^{'}(x) &= b_{n-1} + 2c_{n-1}(x-x_{n-1}) + 3d_{n-1}(x-x_{n-1})^2\\
        &\newline\\
        {S_0}^{''}(x) &= 2c_0 + 6d_0(x-x_0)\\
        {S_1}^{''}(x) &= 2c_1 + 6d_1(x-x_1)\\
        &\vdots\\
        {S_{n-2}}^{''}(x) &= 2c_{n-2} + 6d_{n-2}(x-x_{n-2})\\
        {S_{n-2}}^{''}(x) &= 2c_{n-1} + 6d_{n-1}(x-x_{n-1})
    \end{align}
    Using this information, we further know that the cubic polynomials follow the conditions by matching the first and second derivatives of the inner nodes of two polynomials:
    \begin{align}
        {S_0}^{'}(x_1) = {S_1}^{'}(x_1) &\Rightarrow b_0 + 2c_0(x_1-x_0) + 3d_0(x_1-x_0)^2 = b_1\\
        &\Rightarrow b_0 + 2c_0(x_1-x_0) + 3d_0(x_1-x_0)^2 - b_1 = 0\\
        {S_1}^{'}(x_2) = {S_2}^{'}(x_2) &\Rightarrow b_1 + 2c_1(x_2-x_1) + 3d_1(x_2-x_1)^2 = b_2\\
        &\Rightarrow b_1 + 2c_1(x_2-x_1) + 3d_1(x_2-x_1)^2 - b_2 = 0\\
        &\vdots\\
        {S_{n-3}}^{'}(x_{n-2}) = {S_{n-2}}^{'}(x_{n-3}) &\Rightarrow b_{n-3} + 2c_{n-3}(x_{n-2}-x_{n-3}) + 3d_{n-3}(x_{n-2}-x_{n-3})^2 = b_{n-2}\\
        &\Rightarrow b_{n-3} + 2c_{n-3}(x_{n-2}-x_{n-3}) + 3d_{n-3}(x_{n-2}-x_{n-3})^2 - b_{n-2} = 0\\
        {S_{n-2}}^{'}(x_{n-1}) = {S_{n-1}}^{'}(x_{n-1}) &\Rightarrow b_{n-2} + 2c_{n-2}(x_{n-1}-x_{n-2}) + 3d_{n-2}(x_{n-1}-x_{n-2})^2 = b_{n-1}\\
        &\Rightarrow b_{n-2} + 2c_{n-2}(x_{n-1}-x_{n-2}) + 3d_{n-2}(x_{n-1}-x_{n-2})^2 - b_{n-1} = 0\\
        &\newline\\
        {S_0}^{''}(x_1) = {S_1}^{''}(x_1) &\Rightarrow 2c_0 + 6d_0(x_1-x_0) = 2c_1 \Rightarrow 2c_0 + 6d_0(x_1-x_0) - 2c_1 = 0\\
        {S_1}^{''}(x_2) = {S_2}^{''}(x_2) &\Rightarrow 2c_1 + 6d_1(x_2-x_1) = 2c_2 \Rightarrow 2c_1 + 6d_1(x_2-x_1) - 2c_2 = 0\\
        &\vdots\\
        {S_{n-3}}^{''}(x_{n-2}) = {S_{n-2}}^{''}(x_{n-3}) &\Rightarrow 2c_{n-3} + 6d_{n-3}(x_{n-2}-x_{n-3}) = 2c_{n-2}\\
        &\Rightarrow 2c_{n-3} + 6d_{n-3}(x_{n-2}-x_{n-3}) - 2c_{n-2} = 0\\
        {S_{n-2}}^{''}(x_{n-1}) = {S_{n-1}}^{''}(x_{n-1}) &\Rightarrow 2c_{n-2} + 6d_{n-2}(x_{n-1}-x_{n-2}) = 2c_{n-1}\\
        &\Rightarrow 2c_{n-2} + 6d_{n-2}(x_{n-1}-x_{n-2}) - 2c_{n-1} = 0
    \end{align}
    THe above gives us $2(n-1)$ equations, and the not-a-knot condition gives us the last two equations. To use the not-a-knot condition, we need to first compute the third order derivatives of the first two and the last two polynomials:
    \begin{align}
        {S_0}^{'''}(x) &= 6d_0\\
        {S_1}^{'''}(x) &= 6d_1\\
        {S_{n-2}}^{'''}(x) &= 6d_{n-2}\\
        {S_{n-1}}^{'''}(x) &= 6d_{n-1}
    \end{align}
    Therefore, the not-a-knot condition gives us the equation:
    \begin{align}
        {s_0}^{'''}(x_1) = {s_1}^{'''}(x_1) &\Rightarrow 6d_0 = 6d_1 \Rightarrow d0 = d1\\
        {s_{n-1}}^{'''}(x_{n-1}) = {s_{n-2}}^{'''}(x_{n-1}) &\Rightarrow 6d_{n-2} = 6d_{n-1} \Rightarrow d_{n-2} = d_{n-1}
    \end{align}
    Thus, in total, we have $2n + 2(n-1) + 2 = 4n$ equations to solve for $4n$ coefficients for the n polynomials. The system of equations is set up with equations (33) - (45), (57) - (73) and (78) - (79).
\end{answer}