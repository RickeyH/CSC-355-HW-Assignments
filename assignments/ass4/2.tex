\section{Question 2}

\begin{question}
    Construct a natural cubic spline interpolating the data given by the table \begin{tabular}{c|c} $x$ & $f(x)$ \\ \hline 0.1 & 0.29 \\ 0.2 & 0.56 \\ 0.3 & 0.81 \end{tabular}
\end{question}

\begin{answer}
    First, we could construct two cubic polynomials:
    \begin{align}
        S_0(x) &= a_0 + b_0(x-0.1) + c_0(x-0.1)^2 + d_0(x-0.1)^3\\
        S_1(x) &= a_1 + b_1(x-0.2) + c_1(x-0.2)^2 + d_1(x-0.2)^3
    \end{align}
    Then, these two polynomials need to satisfy the following conditions:
    \begin{align}
        S_0(0.1) = f(0.1) &\Rightarrow a_0 = 0.29\\
        S_0(0.2) = f(0.2) &\Rightarrow a_0 + 0.1b_0 + 0.01c_0 + 0.001d_0 = 0.56\\
        S_1(0.2) = f(0.2) &\Rightarrow a_1 = 0.56\\
        S_1(0.3) = f(0.3) &\Rightarrow a_1 + 0.1b_1 + 0.01c_1 + 0.001d_1 = 0.81
    \end{align}
    Computing the first and second polynomials of the polynomials, we have
    \begin{align}
        {S_0}^{'}(x) &= b_0 + 2c_0(x-0.1) + 3d_0(x-0.1)^2\\
        {S_0}^{''}(x) &= 2c_0 + 6d_0(x-0.1)\\
        {S_1}^{'}(x) &= b_1 + 2c_1(x-0.2) + 3d_1(x-0.2)^2\\
        {S_1}^{''}(x) &= 2c_1 + 6d_1(x-0.2)
    \end{align}
    Using this information, we further know that the cubic polynomials follow the conditions by matching the first and second derivatives of the inner nodes of two polynomials:
    \begin{align}
        {S_0}^{'}(0.2) = {S_1}^{'}(0.2) &\Rightarrow b_0 + 0.2c_0 + 0.03d_0 = b_1 \Rightarrow b_0 + 0.2c_0 + 0.03d_0 - b_1 = 0\\
        {S_0}^{''}(0.2) = {S_1}^{''}(0.2) &\Rightarrow 2c_0 + 0.6d_0 = 2c_1 \Rightarrow 2c_0 + 0.6d_0 - 2c_1 = 0\\
        {S_0}^{''}(0.1) = 0 &\Rightarrow 2c_0 = 0\\
        {S_1}^{''}(0.3) = 0 &\Rightarrow 2c_1 + 0.6d_1 = 0
    \end{align}
    Using the above conditions, we can set up a system equation in an augmented matrix form:
    
    \begin{center}
    $\left[\begin{matrix}
        1 & 0 & 0 & 0 & 0 & 0 & 0 & 0 & 0.29\\
        1 & 0.1 & 0.01 & 0.001 & 0 & 0 & 0 & 0 & 0.56\\
        0 & 0 & 0 & 0 & 1 & 0 & 0 & 0 & 0.56\\
        0 & 0 & 0 & 0 & 1 & 0.1 & 0.01 & 0.001 & 0.81\\
        0 & 1 & 0.2 & 0.03 & 0 & -1 & 0 & 0 & 0 \\
        0 & 0 & 2 & 0.6 & 0 & 0 & -2 & 0 & 0\\
        0 & 0 & 2 & 0 & 0 & 0 & 0 & 0 & 0\\
        0 & 0 & 0 & 0 & 0 & 0 & 2 & 0.6 & 0
    \end{matrix}\right]$
    \end{center}
    
    In \MATLAB, I used the following codes to compute the solution of the matrix equation
    \begin{verbatim}
        % Find the solution to the system of equations constructed
        A = [1,0,0,0,0,0,0,0;
             1,0.1,0.01,0.001,0,0,0,0;
             0,0,0,0,1,0,0,0;
             0,0,0,0,1,0.1,0.01,0.001;
             0,1,0.2,0.03,0,-1,0,0;
             0,0,2,0.6,0,0,-2,0;
             0,0,2,0,0,0,0,0;
             0,0,0,0,0,0,2,0.6];
        b = [0.29;0.56;0.56;0.81;0;0;0;0];
        % Using the backslash to solve the matrix equation
        x0 = A\b;
        % Using the inv function to solve the matrix equation
        x = inv(A)*b;
    \end{verbatim}
    I tried two ways in \MATLAB to find the solution, which are $\backslash$ command and the \textit{inv()} function. The results given by the both methods are almost the same except the entry for $c_0$. The $\backslash$ method give a result of $-1.11022302462518e^{-17} \approx 0$, while the \verb+inv()+ method gives a result of $0$. I prefer the result given by the \verb+inv()+ method, since we can see that $c_0 = 0$ by the second last row of the augmented matrix.
    Hence the solution is the vector
    \begin{center}
        $\left[\begin{matrix}
        0.29\\
        2.75\\
        0\\
        -5.00\\
        0.56\\
        2.60\\
        -1.50\\
        5.00
        \end{matrix}\right]$
    \end{center}
    Therefore, the natural spline interpolating the data is
    \begin{align}
        S_0(x) &= 0.29 + 2.75(x-0.1) - 5(x-0.1)^3\\
        S_1(x) &= 0.56 + 2.6(x-0.2) - 1.5(x-0.2)^2 + 5(x-0.2)^3
    \end{align}
\end{answer}