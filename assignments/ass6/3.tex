\section{Question 3}

\begin{question}
    Approximate the following integrals in \MATLAB using Gaussian quadrature via the function \verb+qrule.m+ (downloaded from Canvas) with $n = 1,2,3,4,5$. You'll need to write a change of variables function \verb+function t = varchange(x,a,b)+ that transforms the points in $[-1,1]$ to the desired interval $[a.b]$. Use both the Gauss-Legendre rule (using \verb+wfun = 1+) and the Gauss-Chebyshev rule (\verb+wfun = 3+) to approximate the integrals. (Note: the Gauss-Chebyshev rule uses the roots of the Chebyshev polynomials as nodes). Create tables for each integral of the form \begin{tabular}{c|c|c} $n$ & $I_L$ & $I_C$ \\ \hline && \end{tabular} where $I_L$ represents the integral value for Gauss-Legendre, and $I_C$ represents the integral value for Gauss-Chebyshev. Compare your answers to the true value given and discuss the best choice of Gauss quadrature rule and $n$ for each integral.
\end{question}

\subsection{Part a}

\begin{question}
    \begin{equation}
        \int_{-2}^3 e^{-x^2}\, dx\text{, true value }\; 1.7683
    \end{equation}
\end{question}

\begin{answer}
    The function \verb+varchange+ I created is:
    \begin{verbatim}
        %% The function varchange
        % The input are the point in [-1,1] x, the target interval left boundary a,
        % and right boundry b. The output is a value in the interval [a,b].
        function t = varchange(x,a,b)
            rate = (b-a)/2;
            t = a + (x+1)*rate;
        end
    \end{verbatim}
    Implementing both function \verb+varchange+ and \verb+qrule+, I have the results
    \begin{table}[H]
    \centering
    \caption{Question 3 Part a Results}
    \label{tab:tab1}
    \begin{tabular}{|c|c|c|}
    \hline
    \textbf{n}              & \textbf{$I_L$}                            & \textbf{$I_C$}                            \\ \hline
    1                       & 3.894003915357025                      & 3.894003915357025                      \\ \hline
    2                       & 1.083927661559878                      & 1.541383617787206                      \\ \hline
    3                       & 1.910736299951195                      & 2.017172523137550                      \\ \hline
    \multicolumn{1}{|l|}{4} & \multicolumn{1}{l|}{1.763348640147853} & \multicolumn{1}{l|}{1.923621647921213} \\ \hline
    \multicolumn{1}{|l|}{5} & \multicolumn{1}{l|}{1.758513155277719} & \multicolumn{1}{l|}{1.882891106493818} \\ \hline
    \end{tabular}
    \end{table}
    From the result in Table \ref{tab:tab1}, we can see that when $n = 4$, using the Gauss-Legendre rule, we have the approximation closest to the true value.
\end{answer}

\subsection{Part b}

\begin{question}
    \begin{equation}
        \int_0^4 \frac{1}{(x-5)^2}\, dx\text{, true value}\; 0.8
    \end{equation}
\end{question}

\begin{answer}
     Implementing both function \verb+varchange+ and \verb+qrule+, I have the results
    \begin{table}[H]
    \centering
    \caption{Question 3 Part b Results}
    \label{tab:tab2}
    \begin{tabular}{|c|c|c|}
    \hline
    \textbf{n}              & \textbf{$I_L$}                            & \textbf{$I_C$}                            \\ \hline
    1                       & 0.444444444444444                      & 0.444444444444444                      \\ \hline
    2                       & 0.703213610586011                      & 0.625000000000000                      \\ \hline
    3                       & 0.779104173043567                      & 0.696397077349458                      \\ \hline
    \multicolumn{1}{|l|}{4} & \multicolumn{1}{l|}{0.795968019037723} & \multicolumn{1}{l|}{0.726420515206909} \\ \hline
    \multicolumn{1}{|l|}{5} & \multicolumn{1}{l|}{0.799269019735260} & \multicolumn{1}{l|}{0.741465185556776} \\ \hline
    \end{tabular}
    \end{table}
    From the result in Table \ref{tab:tab2}, we can see that when $n = 5$, using the Gauss-Legendre rule, we have the approximation closest to the true value.
\end{answer}

\subsection{Part c}

\begin{question}
    \begin{equation}
        \int_{-3}^{-1} \sqrt{x+5}\, dx\text{, true value}\; 3.4478
    \end{equation}
\end{question}

\begin{answer}
    Implementing both function \verb+varchange+ and \verb+qrule+, I have the results
    \begin{table}[H]
    \centering
    \caption{Question 3 Part c Results}
    \label{tab:tab3}
    \begin{tabular}{|c|c|c|}
    \hline
    \textbf{n}              & \textbf{$I_L$}                            & \textbf{$I_C$}                            \\ \hline
    1                       & 3.464101615137754                      & 3.464101615137754                      \\ \hline
    2                       & 3.447874791455152                      & 3.451967523471161                      \\ \hline
    3                       & 3.447717776461481                      & 3.450497984205399                      \\ \hline
    \multicolumn{1}{|l|}{4} & \multicolumn{1}{l|}{3.447715298286575} & \multicolumn{1}{l|}{3.449882572785901} \\ \hline
    \multicolumn{1}{|l|}{5} & \multicolumn{1}{l|}{3.447715251181106} & \multicolumn{1}{l|}{3.449511597449320} \\ \hline
    \end{tabular}
    \end{table}
    From the result in Table \ref{tab:tab3}, we can see that when $n = 2$, using the Gauss-Legendre rule, we have the approximation closest to the true value.
\end{answer}