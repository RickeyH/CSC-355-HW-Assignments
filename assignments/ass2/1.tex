\section{Question 1}

\begin{question}
    For a function $f(x)$ with continuous first and second derivatives, prove that $f$ has a zero $p$ of multiplicity 2 if and only if $0 = f(p) = f'(p)$ but $f''(p) \neq 0$.
\end{question}

\begin{answer}
    \begin{proof}
        By the root theorem, we know that $x - p$ is a factor of $f(x)$ and $f'(x)$ since $f(p) = f'(p) = 0$. However, $x - p$ is not a factor of $f''(x)$ because $f''(p) \neq 0$. These means that $x - p$ has degree two in the most factorized form of $f(x)$, because, each time we differentiate $f(x)$, the power of $x - p$ in the factorization will lower by one. Thus, due that the degree of $x - p$ in the factorization of $f(x)$ is $2$, we know that the zero $p$ of $f(x)$ is of multiplicity of $2$ by the root theorem.
    \end{proof}
\end{answer}